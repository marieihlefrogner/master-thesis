\section{Classification CNN}
\begin{figure}[!ht]
\begin{code}
  \begin{minted}[linenos]{python}
  def create_classification_model(L, n_output_layers):
    model = Sequential()

    model.add(Reshape((L, 1), input_shape=(L,)))
    model.add(Conv1D(100, 10, activation='relu', input_shape=(L, 1)))
    model.add(Conv1D(100, 10, activation='relu'))
    model.add(MaxPooling1D(2))
    model.add(Conv1D(160, 10, activation='relu'))
    model.add(Conv1D(160, 10, activation='relu'))
    model.add(GlobalAveragePooling1D())
    model.add(Dropout(0.5))
    model.add(Dense(n_output_layers, activation='softmax'))

    model.compile(loss='categorical_crossentropy', 
                  optimizer='adam', 
                  metrics=['accuracy'])

    return model
  \end{minted}
  \caption{1D CNN Model for Classification. A total of four convolutional layers together with pooling compiled with Categorical Crossentropy enables the model to classify.}
  \label{code:1d_conv_net_classifier}
\end{code}
\end{figure}

\newpage

\section{Prediction CNN}
\begin{figure}[!ht]
\begin{code}
  \begin{minted}[linenos]{python}
  def create_prediction_model(L):
    model = Sequential()

    model.add(Reshape((L, 1), input_shape=(input_shape,)))
    model.add(Conv1D(128, 2, activation='relu', input_shape=(L, 1)))
    model.add(MaxPooling1D(pool_size=2, strides=1))
    model.add(Conv1D(64, 2, activation='relu'))
    model.add(GlobalAveragePooling1D())
    model.add(Flatten())
    model.add(Dense(10, activation='relu'))
    model.add(Dense(1))
    
    model.compile(loss='mean_squared_error', 
                  optimizer='adam', 
                  metrics=['mse'])

    return model
  \end{minted}
  \caption{1D CNN Model for Prediction. The layers are changed to be able to predict a value instead of performing classification. The model is compiled with the loss function Mean Squared Error, and uses the optimizer Adam}
  \label{code:1d_conv_net_predictor}
\end{code}
\end{figure}